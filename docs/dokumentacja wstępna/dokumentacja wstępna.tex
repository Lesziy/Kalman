\documentclass{article}

\usepackage{amssymb}
\usepackage{polski}
\usepackage[polish]{babel}
\usepackage[utf8]{inputenc}
\usepackage{amsmath}
\usepackage{hyperref}
\usepackage{graphicx}

\title{Aplikacja do śledzenia celów i fuzji danych  \\ {\large Dokumentacja wstępna}}
\author{Przemysław Saramonowicz, Marcin Buczko, Jacek Palczewski \\ Wydział Elektroniki i Technik Informacyjnych \\ Politechnika Warszawska}
\date{28 października 2015 r.}
%wypełnienie strony
%\setlength{\textheight}{24cm}
%\setlength{\textwidth}{15.92cm}
%\setlength{\footskip}{10mm}
%\setlength{\oddsidemargin}{0mm}
%\setlength{\evensidemargin}{0mm}
%\setlength{\topmargin}{0mm}
%\setlength{\headsep}{5mm}

%kropki po \section
\usepackage{titlesec}
\titlelabel{\thetitle.\quad}

\begin{document}
	\maketitle
	
	\section{Zagadnienie}
		W ramach projektu stworzona zostanie aplikacja, która na podstawie symulowanych danych o położeniu obiektu, będzie badała jego położenie, za pomocą sensorów, a następnie badała położenie i przewidywała tor ruchu obiektu w kolejnych iteracjach oraz porównywała tor ruchu przekazany przez obiekt i po pomiarze a następnie określała błąd przewidywania toru. 
		
	\section{Założenia projektowe}
		Projekt zostanie podzielony na 5 podstawowych modułów:
		
		\begin{itemize}
		\item \textbf{Symulacja ruchu obiektu} - obiekt będzie poruszał się w przestrzeni dwuwumiarowej po zdefiniowanych wcześniej trajektoriach ruchu oraz przekazywał dane o swoim położeniu sensorom.

		\item \textbf{Sensor ruchu} - odbiera dane o położeniu obiektu i przekazuje je razem z własnym szumem pomiarowym na wejście filtru Kalmana.
		
		\item \textbf{Filtr Kalmana} - Odbiera dane z sensorów i wyznacza stan wewnętrzny modelowanego obiektu
		
		\item \textbf{Moduł określający błąd pomiaru} - Odbiera dane od obiektu, sensorów oraz filtru Kalmana i określa jak dokładnie filtr Kalmana przewiduje tor w kolejnych iteracjach oraz określa poprawę w stosunku do odczytów sensorów.
		
		\item \textbf{Interpretacja graficzna działania programu} - Rysuje obiekt w kolejnych iteracjach na jego pozycji oraz jego kopię w miejscu określonym przez układ sensorów z filtrem Kalmana a także trajektorię obu instancji obiektu.
		\end{itemize}		
		
W wyniku działania programu zostanie zbadany wpływ ilości sensorów oraz szumów na dokładność odczytu. Będzie on również badał wpływ toru ruchu oraz zmienności prędkości na dokładność pomiaru.


Aplikacja będzie spełniała założenia przenośności tzn. będzie ją można uruchomić na różnych dystrybucjach Linuxa oraz na systemach z rodziny Windows.
				

%\begin{thebibliography}{19}
 
%\bibitem{regulamin}
%  Centralne Laboratorium Fizyki Wydziału Fizyki Politechniki Warszawskiej, \\
%  \emph{Regulamin CLF}.
%  Dostęp: 21 marca, 2015 r,   
%  \url{http://clf.if.pw.edu.pl/?q=pl/regulamin-clf}
  
%\end{thebibliography}


\end{document}