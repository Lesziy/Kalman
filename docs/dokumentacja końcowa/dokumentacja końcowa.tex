\documentclass{article}

\usepackage{amssymb}
\usepackage{polski}
\usepackage[polish]{babel}
\usepackage[utf8]{inputenc}
\usepackage{amsmath}
\usepackage{hyperref}
\usepackage{graphicx}

\title{Aplikacja do śledzenia celów i fuzji danych  \\ {\large Dokumentacja końcowa}}
\author{Przemysław Saramonowicz, Marcin Buczko, Jacek Palczewski \\ Wydział Elektroniki i Technik Informacyjnych \\ Politechnika Warszawska}
\date{26 stycznia 2015 r.}
%wypełnienie strony
%\setlength{\textheight}{24cm}
%\setlength{\textwidth}{15.92cm}
%\setlength{\footskip}{10mm}
%\setlength{\oddsidemargin}{0mm}
%\setlength{\evensidemargin}{0mm}
%\setlength{\topmargin}{0mm}
%\setlength{\headsep}{5mm}

%kropki po \section
\usepackage{titlesec}
\titlelabel{\thetitle.\quad}

\begin{document}
	\maketitle
	
	\section{Realizacja zagadnień}
		Podsumowanie zostanie oparte o punkty z dokumentacji wstępnej, które zostaną skomentowane na podstawie bieżących etapów prac.	
		\begin{itemize}
		\item \textbf{Symulacja ruchu obiektu} - obiekt będzie poruszał się w przestrzeni dwuwumiarowej po zdefiniowanych wcześniej trajektoriach ruchu oraz przekazywał dane o swoim położeniu sensorom.
		\begin{itemize}
			\item Ten moduł programu udało się zrealizować w zadowalającym stopniu: trajektoria obiektu jest generowana w oparciu o skrypty w języku Python(katalog \texttt{maps/}).
		\end{itemize}

		\item \textbf{Sensor ruchu} - odbiera dane o położeniu obiektu i przekazuje je razem z własnym szumem pomiarowym na wejście filtru Kalmana.
			\begin{itemize}
				\item Miejsce na komentarze... 
			\end{itemize}

		\item \textbf{Filtr Kalmana} - Odbiera dane z sensorów i wyznacza stan wewnętrzny modelowanego obiektu
			\begin{itemize}
				\item Miejsce na komentarze... 
			\end{itemize}
		
		\item \textbf{Moduł określający błąd pomiaru} - Odbiera dane od obiektu, sensorów oraz filtru Kalmana i określa jak dokładnie filtr Kalmana przewiduje tor w kolejnych iteracjach oraz określa poprawę w stosunku do odczytów sensorów.
			\begin{itemize}
				\item Moduł został połączony z poniższym, por. punkt niżej i dalsze rozdziały.
			\end{itemize}
				
		\item \textbf{Interpretacja graficzna działania programu} - Rysuje obiekt w kolejnych iteracjach na jego pozycji oraz jego kopię w miejscu określonym przez układ sensorów z filtrem Kalmana a także trajektorię obu instancji obiektu.
			\begin{itemize}
				\item Mimo pierwotnych planów o zrealizowaniu tego modułu w języku C++, ze względu na problemy z implementacją wyświetlania wyników symulacji w czasie rzeczywistym Zespół podjął decyzję o zmianie podejścia: program składający się z pozostałych modułów jest zrealizowany jako aplikacja w C++, a bieżący moduł został napisany jako uniwersalny skrypt do środowiska Octave(kompatybilnego z Matlabem). Dokładne działanie całego projektu zostanie opisane później.
			\end{itemize}
			
		\item Zależność pomiędzy torem ruchu i prędkością a dokładnością pomiaru; wpływ ilości sensorów na dokładność odczytu. 
			\begin{itemize}
				\item Możliwość uruchomienia z innym torem obiektu(jak również zmiany współczynników - pliki .ini) w opinii Zespołu realizuje olbrzymią część tego założenia.
				\item Miejsce na komentarze... 
			\end{itemize}
		\item Przenośność pomiędzy Windowsem a Linuxem.
			\begin{itemize}
				\item Był to podpunkt intrygujący ze względu na wiele niespodzianek(głównie ze względu na ciekawe różnice pomiędzy kompilatorami(Visual Studio jest o wiele bardziej liberalny i różni się subtelnościami w implementacji niektórych rzeczy nawet w bibliotece standardowej: dla przykładu konstruktor \texttt{std::exception} przyjmujący ciągi znaków pod MSVC, który nie jest dozwolony w GCC - tam identyczną funkcję pełni \texttt{std::runtime\_error})), aczkolwiek dzięki wykorzystaniu Jenkinsa udało się utrzymać ten podpunkt. 
			\end{itemize}
				
		\end{itemize}		
	
	\section{Opis stanu prac nad projektem}
	
	\subsection{Rozwiązania zastosowane w projekcie - synchronizacja, etc.}
		\subsubsection{Klasa Worker}
		 Problem synchronizacji dotyczył praktycznie każdego modułu w aplikacji opartej na wątkach, dlatego został rozwiązany globalnie, poprzez klasę \texttt{Worker<T>}, która korzysta z dobrodziejstw programowania generycznego, czyli trejtów, uproszczając całą obsługę synchronizacji do przeciążenia funkcji \texttt{ThreadProc} w module dziedziczącym po w.w. klasie, a obsługa zmiennej warunkowej, kolejki i synchronizacji jest w gestii kompilatora dla wielu różnych, różniących się szczegółami klas.
	\subsection{Jak program działa - opis}
	\begin{enumerate}
		\item Aplikację można uruchomić z konsoli, bądź poprzez skrypty \texttt{runme}, dzięki czemu 
		\item Generator - zrobię to później
		\item Sensor - krótki opis co klasy robią
		\item Kalman - j.w.
		\item Writer - j.w.
		\item Potencjalne skrypty matlaba.
	\end{enumerate}
	
	\section{Napotkane problemy, popełnione błędy, wnioski}
	
	\subsection{Problemy}
	La, la.
	\subsection{Błędy}
	\subsubsection{GCC i argument określany przez typedef}
	Tutaj Zespół spotkał się z dość dziwnym błędem - GCC nie pozwalało na określanie argumentu funkcji jako typ \texttt{void} w przeciwieństwie do MSVC  który zezwala na taką operację, dlatego trejt \texttt{CommonUtil::Traits::InputWorker} definiuje typ argumentu funkcji \texttt{ThreadProc} jako int.  
	\subsection{Wnioski}
 	4 5 6.
 	
 	\section{Końcowe statystyki}
 	Liczba linii kodu, liczba commitów z gita, liczba testów i pomiar, odwołanie się do nieudanej próby z gcovem.
 	
%\begin{thebibliography}{19}
 
%\bibitem{regulamin}
%  Centralne Laboratorium Fizyki Wydziału Fizyki Politechniki Warszawskiej, \\
%  \emph{Regulamin CLF}.
%  Dostęp: 21 marca, 2015 r,   
%  \url{http://clf.if.pw.edu.pl/?q=pl/regulamin-clf}
  
%\end{thebibliography}


\end{document}